\chapter{关于本书的对话}
\begin{mydlg}
	\textbf{教授:}欢迎来到本书!这本书名叫\textbf{操作系统:Three Easy Pieces},我在这
	是来教你有关操作系统的东西的。我叫“教授”,你是?
	
	\textbf{学生:}你好教授!你也许猜到了,我是学生。我已经准备好学习了!
	
	\textbf{教授:}很好。有任何问题吗?
	
	\textbf{学生:}当然!为什么叫做“Three Easy Pieces”?
	
	\textbf{教授:}这个简单。你看这有费曼的一份极棒的物理学讲义\ldots
	
	\textbf{学生:}哦!那个写了“别闹了,费曼先生”的人,对吧?确实是好书。那这本书也会想
	那本一样幽默吗?
	
	\textbf{教授:}恩\ldots 不会。那本书是很棒,我很高兴你读过它。希望这本书更像他的物理
	学笔记。其中的一些基础总结起来了,称为“Six Easy Pieces”。他谈论的是物理学;而我们会
	讲操作系统相关的“Three Easy Pieces”。这就好像说操作系统大概有物理学一半那么难。
	
	\textbf{学生:}我喜欢物理学,所以应该挺好的。有哪些小块呢?
	
	\textbf{教授:}他们是我们将要学的三个关键思想:虚拟化,并发性和持久化。在学习这些思
	想的过程中,我们会学到所有关于操作系统如何工作的知识,包括如何决定下一个在CPU上运行
	的程序,在虚拟内存系统中如何处理内存过载,虚拟机监视器是怎么工作的,如果管理磁盘信
	息,甚至还有一些怎么构建一个当部分失效时仍然能工作的分布式系统。就是像这样的东西。
	
	\textbf{学生:}我完全不知道你在说些什么,真的。
	
	\textbf{教授:}很好,这说明你上对课了。
	
	\textbf{学生:}我还有一个问题:学这些东西的最好方法是什么?
	
	\textbf{教授:}好问题!每个人都应该自己想出这个问题的答案。但这是我会做的:上一门
	课,听听教授对这些材料的介绍。然后,不如说每周末吧,阅读这些笔记,让自己对这些思想有
	更深刻的印象。当然,过一段时间后(提示:在考试前!),再次阅读这些笔记巩固你的知识。毫
	无疑问,你的教授也会布置一些家庭作业和工程,你应该完成它们;在写实际代码解决实际问题
	的做工程过程是将笔记中的思想付诸实践的最佳方式。就像孔子说过\ldots
	
	\textbf{学生:}啊我知道!‘我听我往,我看我记,我做我理解’或像这样的什么吧。
	
	\textbf{教授:}(惊奇)你怎么知道我要说什么?
	
	\textbf{学生:}那看起来很搭。另外,我是孔子的粉丝。
	
	\textbf{教授:}我想我们这一路上会处的很好的。
	
	\textbf{学生:}教授,还有一个问题。这些对话有什么用呢?我的意思是,这不是一本书吗?为
	什么吧直接放上写材料呢?
	
	\textbf{教授:}恩,好问题,真是好问题!我想适当的把你自己拉出书本叙述有时是有好处
	的;这些对话就是那适当的时候。所以你和我将会一起把所有这些相当复制的思想搞明白的。你
	赞同吗?
	
	\textbf{学生:}所以我们必须要思考?恩,我同意。我的意思是,要不然我还要做什么呢?在
	本书之外我好像又没什么其他事。
	
	\textbf{教授:}可惜我也是。让我们开始吧!
	
\end{mydlg}

