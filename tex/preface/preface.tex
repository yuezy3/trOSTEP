\chapter*{前言}
\phantomsection
\addcontentsline{toc}{section}{给所有人}
\section*{给所有人}
欢迎来到本书。我们希望你会像我们享受写作本书一样享受阅读它。
这本书叫做\mkidx{Operating Systems in Three Easy Pieces}{操作系统Three Easy Pieces},
书名显然是向费曼在物理学上的一部有史以来最伟大的讲义\citeholder{F96}致敬的。
虽然本书毫无疑问达不到像那位著名物理学家的高水准,但也许它对寻求理解整个操作系统(或更
广义地,整个系统)的你来说也许是足够的。

三个简单部分指本书围绕的三个主要专题:
\mkidx{virtualization}{虚拟化},
\mkidx{concurrency}{并发性}和
\mkidx{persistence}{持久化}。在讲这些概念时,我们会讨论大多数
操作系统执行的重要事情;希望你在这过程中也会感到有趣。学习新东西应该很有趣对吧,至少应该
有趣。

每个主要的概念都被分为好几个章节,其中大部分章节是提出一个具体问题然后展示如何解决它。
每章都很短,并且尽量参考那些最初提供那些想法(概念的原型)的材料。我们写作本书的一个
目标就是使历史的路径尽可能清晰,因为这样将帮助学生更清楚的理解“它现在是什么、它曾
经是什么、它将来会是什么”。 在这里,知道香肠是怎么制作的和理解香肠适合做什么同样重
要\footnote{提示:吃!或者如果你是一个素食主义者,走开}。

有一些本书通篇会使用的方法值得在这里提一下。第一个就是问题的
\mkidx{crux}{症结}。任何时候
我们试图解决一个问题时,首先就是尝试陈述问题最重要的焦点是什么;在行文中
想这样的\mkidx{crux of the problem}{问题的症结}
会被直接指出,而且希望通过后面呈现的一些技术、算法和想法来解决。

文中也有许多为正文提供一些色彩的\mkidx{asides}{旁白}和
\mkidx{tips}{提示}。旁白通常讨论和正文相关的东西(但可能不是本质的);
提示通常是可以应用到系统的一般经验。为了方便,本书末尾有一个列出了所有旁白、提示和症结的
索引。

我们在本书中还使用了一种古老的说教方法:\mkidx{dialogue}{对话},作为
呈现一些材料的不同方式。对话被用来引入主要概念(以一种很自然的方式,你会看到)以及时不时
回顾一下材料。对话也可以作为写的更幽默一些的机会,不够你发现它是有用还是幽默,恩,这是另
一回事了。

在每个主要部分的开始处,我们首先给出操作系统所提供的
\mkidx{abstraction}{抽象},然后在后续章节里探索为提供这个抽象
所需要的机制、策略和其他需要的支持。抽象是计算机科学所有方面的基础,所以也毫无疑问是操作系统
的本质。

在各个章节里,我们在可能的地方都使用\mkidx{real code}{实际代码}
(而不是\mkidx{pseudocode}{伪代码}),所以基本上
所有例子,你都可以能够输入和运行它们。在真实系统上运行实际代码会是学习操作系统最好的方法。所
以我们也鼓励你可以这样做时就去做。

在本书的不同地方,我们都搞了点\mkidx{homeworks}{家庭作业}来保证
你理解文中内容。许多这样的家庭作业都是
操作系统部件的小仿真;你应该下载这些作业,运行他们来测验自己。家庭作业的仿真器有一下特征:
通过给他们不同的随机数种子,你几乎可以产生无穷的问题集;仿真器也可以被要求解决你的问题。因
此,你可以测试、再测试自己直到你达到一个比较好的理解程度。

本书最重要的补充是一些\mkidx{projects}{工程},通过设计、实现和测试你自己的代码,你可以学到真实系统是
如何运行的。所有的工程(包括代码示例,前面提到过)用的都是C语言
\mkidx{C programming language}{C语言} \citeholder{KR88};C是大多数
操作系统底层采用的一种简单强大的语言,很值得你将它加入你自己的语言工具箱。有两种类型的工程
可供选择(到在线附录里去)。第一种是
\mkidx{systems programming}{系统编程}类项目;
这种工程很适合想学底层C编程的C和UNIX新
手。第二种是基于MIT开发的一个叫xv6\citeholder{CK+08}的真实操作系统内核;这种工程很适合那些
想深入到操作系统内部的已经具备了C语言基础的同学。在华盛顿,我们以三种不同方式开过这个课:
全是系统编程,全是xv6编程,或者这两者混合。

\clearpage

\phantomsection
\addcontentsline{toc}{section}{给执教者}
\section*{给执教者}
如果你是想使用本书的教师或教授\cite{RichardP.Feynman2011},请自由使用。你也许注意到,它们
可以在下面的网页上自由获取:

%\lstset{numbers=none}
%\begin{lstlisting}
%        http://www.ostep.org
%\end{lstlisting}
\begin{quote}
	\url{http://www.ostep.org}
\end{quote}

你也可以在\url{lulu.com}上买打印的副本。在上面的页面上可以找到。

目前可以像这样来引用本书: 
\begin{quote}
	\textbf{Operating Systems: Three Easy Pieces} \\
	Remzi H. Arpaci-Dusseau and Andrea C. Arpaci-Dusseau \\
	Arpaci-Dusseau Books, Inc. \\
	May, 2014 (Version 0.8) \\
	\url{http://www.ostep.org}
\end{quote}

这个课程分成15周的学期比较好,这样你可以以一个合理的深度讲完大部分主题。将课程压缩到一个
10周的季度可能要丢掉一些细节。也有些章节是关于虚拟机监视器的,我们通常将其挤到一个学期的
不同时候,要么在虚拟化部分结束时,要么在接近结束时作为旁白。

本书一个稍微不同的地方是并发性部分,大多数操作系统的书籍将这个主题放到前面,本书将其推后到
学生已经对CPU和内存的虚拟化有了一定理解之后。教授这门课的将近15年的经验显示,如果不理解
什么是地址空间,什么是进程,或者为什么上下文切换可以发生在任意时间点,那么学生将很难理解
并发问题是怎么产生的,或者为什么他们会尝试去解决并发问题。而一旦他们理解了那些概念,引入
线程记号和由此引发的问题就会变得相对容易,或者至少更容易些。

你也许注意到本书没有附带幻灯片。这个缺失的主要原因是我们相信最老式教学方法:粉笔和一块黑板。
所以,当教授这门课时,我们带着几个主要观念和几个例子,然后用黑板呈现它们;讲义文稿和少许
代码示例也是很有用的。以我们的经验,使用过多的幻灯片会使学生们仅仅“登记”一下课程(然后就去
刷脸书),因为它们知道材料就在那等着他们稍后消化;使用黑板使课具有生动的体验,因而(希望)
对课上的学生来说更具有交互性、动态性和有趣。

如果你想要一份我们备课的笔记,请发封邮件索取。我们已经将其分享给了全世界的很多人了。

最后一个请求:如果你使用在线免费章节,请不要做本地拷贝,直接\textbf{链接}到章节就是。这将
帮助我们追踪使用情况(过去几年有超过一百万次下载!)也会保证学生们得到最新的版本。

\clearpage

\phantomsection
\addcontentsline{toc}{section}{给学生}
\section*{给学生}
如果你是阅读本书的学生,非常感谢!我们很荣幸能提供一些资料帮助你追寻有关操作系统的知识。我们
都深情的回忆起我们本科阶段的一些教科书(例如,Hennessy and Patterson\citeholder{HP90},计
算机架构的经典教科书)并希望本书会成为你具有良好回忆的书中的一本。

你也许注意到本书可以在线免费获取。这样做的一个主要的原因是:教科书通常都太贵了。这本书,我们
希望,会是帮助那些在追寻良好教育的人的第一波免费材料,不管他们是来自世界的那个地方或他们
愿意为书本花多少钱。如果做不到那样,这本免费的书,也总好过没有。

我们同时也希望,在可能的地方,向你指出本书许多材料的最初来源:那些伟大的论文和那些在操作系统
领域耕耘多年的人。想法不是凭空产生的;它们来自那些聪明并且努力工作的人(包括许多图灵奖获
得者\footnote{图灵奖是计算机科学的最高奖项,就像诺贝尔奖,只不过你可能从来没听说过。}),
因此我们在可能时应该努力赞扬那些想法和那些人。这样,我们希望能更好的理解那时发生的变革,而不是
像那些观点从来就存在一样书写\citeholder{K62}。更进一步地,这样的引用文献也许会鼓励你自己
更进一步的深究;阅读本领域内的著名论文的确会是一种最好的学习方法。
\clearpage

\phantomsection
\addcontentsline{toc}{section}{致谢}
\section*{致谢}
本节包含所有帮助过本书写作的人的感谢。最重要的事是:\textbf{你的名字也可以在这里!}但是,你
必须先提供帮助。所以请给我们发送反馈并帮助本书除错。然后你就可以出名了!或者,至少将你的名字
写入一些书里。

到目前为止帮助过我们的人包括:Abhirami Senthilkumaran*, Adam Drescher* (WUSTL), 
Adam Eggum, Ahmed Fikri*, Ajaykrishna Raghavan, Akiel Khan, Alex Wyler, 
Anand Mundada, B. Brahmananda Reddy (Minnesota), Bala SubrahmanyamKambala, 
Benita Bose, BiswajitMazumder (Clemson), Bobby Jack, Bj ¨orn Lindberg, 
Brennan Payne, Brian Kroth, Cara Lauritzen, Charlotte Kissinger, 
Chien-Chung Shen (Delaware)*, Christoph Jaeger, Cody Hanson, 
Dan Soendergaard (U. Aarhus), David Hanle (Grinnell), Deepika Muthukumar, 
Dorian Arnold (NewMexico), DustinMetzler, Dustin Passofaro, Emily Jacobson, EmmettWitchel (Texas), Ernst Biersack (France), Finn Kuusisto*, 
Guilherme Baptista, Hamid Reza Ghasemi, Henry Abbey, Hrishikesh Amur, 
Huanchen Zhang*, Hugo Diaz, Jake Gillberg, James Perry (U.Michigan-Dearborn)*, 
Jan Reineke (Universit¨at des Saarlandes), Jay Lim, Jerod Weinman (Grinnell), 
Joel Sommers (Colgate), Jonathan Perry (MIT), Jun He, Karl Wallinger, 
Kartik Singhal, Kaushik Kannan, Kevin Liu*, Lei Tian (U.Nebraska-Lincoln), 
Leslie Schultz, LihaoWang, MarthaFerris,Masashi Kishikawa (Sony), Matt Reichoff, 
Matty Williams, Meng Huang, Mike Griepentrog, Ming Chen (Stonybrook), 
Mohammed Alali (Delaware),Murugan Kandaswamy, Natasha Eilbert,  Nathan Dipiazza, 
Nathan Sullivan, Neeraj Badlani (N.C. State), Nelson Gomez, NghiaHuynh (Texas), 
Patricio Jara, Radford Smith, RiccardoMutschlechner, Ripudaman Singh, Ross Aiken, Ruslan Kiselev, Ryland Herrick, Samer AlKiswany, SandeepUmmadi (Minnesota), 
Satish Chebrolu (NetApp), Satyanarayana Shanmugam*, Seth Pollen, Sharad Punuganti, 
Shreevatsa R., Sivaraman Sivaraman*, Srinivasan Thirunarayanan*, 
Suriyhaprakhas BalaramSankari, SyJinCheah, Thomas Griebel, Tongxin Zheng, 
Tony Adkins, Torin Rudeen (Princeton), Tuo Wang, Varun Vats, Xiang Peng, Xu Di, 
Yue Zhuo (Texas A\&M), Yufui Ren, Zef RosnBrick, Zuyu Zhang. 特别感谢以上那些标注了
星号的人,他们不仅仅提供建议帮助提高本书。

特别感谢Joe Meehean(Lynchburg)教授和他的每章详细笔记,Jerod Weinman(Grinnell)教授
和他的全班的超赞的小册子,Chien-Chung Shen(Delaware)和他的宝贵详细的阅读和注释。所有
的三人都极大的帮助了作者们改善本书的材料。

此外,还要感谢这些年来上537课的上百位学生。特别要感谢那些鼓励本书从笔记成型为书本的08秋季班
学生(他们特别讨厌没有一本教科书来读--执意的学生们!)他们还给我们足够的赞扬让我们保持前进
(其中在今年的课上还有一个滑稽的评论“哎呀我的天啊,你应该写一本教科书啊!”)。

我们也亏欠了几个上xv6实验课(大部分现已纳入537主课程里了)的勇士。09年春季班: 
Justin Cherniak, Patrick Deline, Matt Czech, Tony Gregerson, Michael Griepentrog, 
Tyler Harter, Ryan Kroiss, Eric Radzikowski, Wesley Reardan, Rajiv Vaidyanathan, 
和 Christopher Waclawik。09年秋季班: Nick Bearson, Aaron  Brown, Alex Bird, 
David Capel, Keith Gould, Tom Grim, Jeffrey Hugo, Brandon Johnson, John Kjell, 
Boyan Li, James Loethen,Will McCardell, Ryan Szaroletta, Simon Tso, and Ben Yule.
10年春季班: Patrick Blesi, Aidan Dennis-Oehling, Paras Doshi, Jake Friedman, 
Benjamin Frisch, Evan Hanson, Pikkili Hemanth, Michael Jeung, Alex Langenfeld, 
Scott Rick, Mike Treffert, Garret Staus, Brennan Wall, Hans Werner, Soo-Young Yang, 
和 Carlos Griffin (almost).

虽然我们的研究生们没有直接帮助本书,但他们教了我们很多关于我们已知的系统。他们在
威斯康星州(Wisconsin)时我们定期与他们交谈,但他们确实是在做实际的工作,而且通过告诉
我们他们在做什么让我们每周都能学到些新东西。这个列表包括共同发表过文章的当前的和以前的学
生;星号标记的是那些在我们的引导下获得了博士学位的人:Abhishek Rajimwale, Ao Ma, 
Brian Forney, Chris Dragga, Deepak Ramamurthi, Florentina Popovici*, 
Haryadi S. Gunawi*, James Nugent, John Bent*, Lanyue Lu, Lakshmi Bairavasundaram*, 
Laxman Visampalli, Leo Arulraj, Meenali Rungta,Muthian Sivathanu*, 
Nathan Burnett*, Nitin Agrawal*, Sriram Subramanian*, Stephen Todd Jones*, 
Suli Yang, Swaminathan Sundararaman*, Swetha Krishnan, Thanh Do, 
Thanumalayan S. Pillai, Timothy Denehy*, Tyler Harter, Venkat Venkataramani, 
Vijay Chidambaram, Vijayan Prabhakaran*, Yiying Zhang*, Yupu Zhang*, Zev Weiss.

最后还要感激Aaron Brown,它多年前(09春节班开始)第一次开这个课,然后开xv6实验课(09秋季班),
还要感激为本课做了差不多两年(10秋季班到12春季班)的助教。他的勤奋工作对工程进展(特别是xv6的
那一块)很有帮助,也让威斯康星州(Wisconsin)数不清的本科和研究生们获得了更好的学习体验。就像
Aaron会说的(以他一贯的简洁风格):“谢了”。
\clearpage

\phantomsection
\addcontentsline{toc}{section}{最后一句}
\section*{最后一句}
叶芝(William Butler Yeats)的名句说“教育并不是填桶,而是点亮一把火。”他是对的,同时
也是错的\footnote{假如他确实说过这话;就像很多名言一样,这个名言的历史也是模糊的。}。你其
实还是要做一些“填桶”工作的,而这些笔记在这里无疑会对你的学业有帮助的;毕竟,当到Google面
试,他们问你一些有关信号量的刁钻问题时,你最好还是要知道什么是信号量,对吧?

但是叶芝的主要观点也是明确的:教育的真实意思是让你对某事物产生兴趣,自动的去学习有关主题的
更多知识,而不仅仅是消化已学的并在课上得个好分数。就像我们其中
的一位父亲(Remzi的父亲, Vedat Arpaci)说过:“学习要超越课堂”。

我们写了这些笔记来激发你对操作系统的兴趣,进一步在这个主题下阅读,和你的教授谈这个领域
中进行的令人兴奋的研究,甚至是亲自参与到研究中去。这是一个很棒的领域,到处都是对计算历史
有着重要而深远的影响,而且令人兴奋又精彩的想法。虽然我们知道这把火不会点燃你们全部,但我们
希望他能点燃你们中的大部分,即使是一些也很值得。因为一旦这把火亮了,那就是你真正能够做一些
伟大事情的时候了。因此教育过程的真正要点就是:前进,去学习很多新的迷人的主题,学习,成熟,
然后最重要的,找到那把能够点燃你的火。

\bigskip
\noindent Andrea and Remzi \\
Married couple \\
Professors of Computer Science at the University of Wisconsin \\
Chief Lighters of Fires, 
hopefully.\footnote{如果这听起来像是我们接受了过去一些纵火犯们的历史的话,那你可能会错意
	了。也许。如果这听起来很勉强,那是因为他就是很勉强,但你要原谅我们。}
\clearpage

\phantomsection
\printbibliography[heading=subbibintoc,title={引用文献}]


